\documentclass[a4paper,12pt]{article}
\usepackage[utf8]{inputenc}
\usepackage[T1]{fontenc}
\usepackage[french]{babel}
\usepackage{geometry}
\geometry{left=1.8cm,right=1.8cm,top=1.3cm,bottom=1.6cm}

\usepackage{lmodern}
\usepackage{xcolor}
\definecolor{titleblue}{RGB}{0,51,140}

\usepackage{hyperref}
\hypersetup{
    colorlinks=true,
    urlcolor=titleblue,
    linkcolor=titleblue,
    citecolor=titleblue,
    pdfborder={0 0 0}
}

\usepackage{titlesec}
\usepackage{enumitem}

\pagestyle{empty}

\titleformat{\section}
  {\color{titleblue}\Large\bfseries\uppercase}{}{0em}{}
  [\color{titleblue}\titlerule]

\titlespacing\section{0pt}{8pt}{4pt}
\setlist[itemize]{leftmargin=*,topsep=2pt,itemsep=1pt}

\begin{document}

\begin{center}
    {\Huge\bfseries\color{titleblue} AISSAOUI Ismail}\\[4pt]
    {\Large Ingénieur IA \& Data Science}\\[6pt]

    {\normalsize
    Email : \href{mailto:ismail.aissaoui.pro@gmail.com}{ismail.aissaoui.pro@gmail.com} \quad
    Tél : +213 660 70 77 96 \\
    Portfolio : \href{https://ismail-aissaoui.vercel.app}{ismail-aissaoui.vercel.app} \quad
    Localisation : Chlef, Algérie \\
    LinkedIn : \href{https://linkedin.com/in/aissaoui-ismail-6a77a92a7}{linkedin.com/in/aissaoui-ismail-6a77a92a7} \quad
    GitHub : \href{https://github.com/i-aissaoui}{github.com/i-aissaoui}
    }
\end{center}

%===========================
\section{RÉSUMÉ PROFESSIONNEL}
Ingénieur IA \& Data Science spécialisé dans les modèles Transformers, le NLP, les systèmes de recommandation, l’optimisation, la vision par ordinateur et le MLOps. Conçoit et déploie des solutions IA de bout en bout pour la HR tech, la mobilité, le retail et l’enseignement. Forte expertise en optimisation de modèles, pipelines ML et alignement des livrables avec les objectifs métier.

%===========================
\section{COMPÉTENCES}
\setlength{\parskip}{-1pt}
{\normalsize
\textbf{Programmation}: Python, TypeScript, JavaScript, Java, SQL \\

\textbf{ML / DL}: PyTorch, TensorFlow, Keras, Scikit-learn, Lightning, ONNX \\

\textbf{Techniques IA}: Transformers (BERT, GPT, RAG), GNN, GAN, XGBoost, LSTM \\

\textbf{Optimisation}: PSO, Algorithme génétique, ACO, Recuit simulé, Optimisation bayésienne \\

\textbf{Web \& API}: React, Next.js, FastAPI, Node.js, TailwindCSS \\

\textbf{Data \& MLOps}: MLflow, Airflow, Kubeflow, DVC, Pandas, NumPy \\

\textbf{Bases de données}: PostgreSQL, MongoDB, MySQL, Neo4j, Cassandra \\

\textbf{Cloud \& DevOps}: Docker, Linux, Git, CI/CD, CUDA, AWS, Grafana \\

\textbf{Soft Skills}: Leadership technique, Communication, Gestion des parties prenantes
}

%===========================
\section{EXPÉRIENCE PROFESSIONNELLE}
\textbf{Stagiaire en ingénierie réseau} \hfill Sep 2024 \\
Algérie Télécom RMC Center — Chlef, Algérie

%===========================
\section{FORMATION}
\textbf{École Nationale Supérieure d’Informatique (ESI-SBA)} \hfill 2021--2026 \\
Diplôme d’ingénieur et master en informatique \\
Spécialisation : Intelligence Artificielle \& Informatique

%===========================
\section{RÉALISATIONS CLÉS}
\begin{itemize}
    \item Développement de modèles Transformer (BERT, GPT, RAG) pour des applications multilingues réelles.
    \item Conception de plateformes d’IA explicable pour la HR tech et les institutions publiques.
    \item Système de recommandation basé sur graphes avec +24\% de précision (60\,000 profils).
    \item Pipelines ML complets : ingestion, validation, entraînement, CI/CD et monitoring.
    \item Vision temps réel (YOLO, ByteTrack) réduisant la latence de 32\%.
\end{itemize}

\newpage

%===========================
\section{PROJETS}
\setlength{\parskip}{13pt}
{\small

\textbf{Plateforme IA de recrutement par GNN (Career Connect)} \hfill 2024--2025 \\
Appariement CV/offres basé sur des graphes et des embeddings relationnels. \\
\textit{Tech : PyTorch Geometric, SBERT, FastAPI, PostgreSQL — Concepts : GNN, Similarité, Embeddings}

\textbf{Aura — Système de trading IA} \hfill 2025 \\
Plateforme de trading crypto combinant prévision LSTM et \textbf{Reinforcement Learning} (Deep Q-Learning) pour optimiser les stratégies d’entrée/sortie à long terme. \\
\textit{Tech : TensorFlow, Keras, FastAPI, Next.js, WebSockets — Concepts : Séries temporelles, RL, DQN}

\textbf{Pipeline NLP avancé multi-optimisation} \hfill 2025 \\
Modération multilingue optimisée via PSO, algorithme génétique et optimisation bayésienne. \\
\textit{Tech : DistilBERT, PyTorch, FastAPI — Concepts : Optimisation, Transformers, Classification}

\textbf{Planificateur universitaire intelligent} \hfill 2025 \\
Génération automatique d’emplois du temps sans conflits à l’aide d’heuristiques hybrides. \\
\textit{Tech : FastAPI, NumPy, Pandas — Concepts : GA, PSO, ACO, Recuit simulé}

\textbf{Mini-GPT (Decoder-only)} \hfill 2025 \\
Implémentation d’un GPT compact avec entraînement contrôlé et génération séquentielle. \\
\textit{Tech : PyTorch, CUDA, AdamW, W\&B — Concepts : Transformers, Language Modeling}

\textbf{North African Sentiment BERT} \hfill 2025 \\
Adaptation LoRA de BERT pour l’analyse de sentiment en dialecte maghrébin. \\
\textit{Tech : PyTorch, LoRA, MLflow — Concepts : Fine-tuning, Low-Rank Adaptation}

\textbf{Geo-RAG — Assistant cartographique} \hfill 2025 \\
Système RAG intégrant des données géospatiales pour répondre à des requêtes cartographiques. \\
\textit{Tech : ChromaDB, GeoPandas, Streamlit — Concepts : RAG, Raisonnement spatial}

\textbf{Recognizini — Reconnaissance faciale} \hfill 2025 \\
Reconnaissance faciale avec embeddings incrémentaux et boucle de feedback actif. \\
\textit{Tech : OpenCV, PyTorch, Redis, FastAPI — Concepts : Embeddings, Drift Management}

\textbf{Détection EPI temps réel (YOLOv8)} \hfill 2025 \\
Détection temps réel des équipements de sécurité pour le suivi de conformité. \\
\textit{Tech : YOLOv8, OpenCV — Concepts : Object Detection, Streaming}

\textbf{Suite analytique trafic} \hfill 2025 \\
Analyse du trafic routier : suivi, estimation de vitesse et OCR des plaques. \\
\textit{Tech : YOLOv8, ByteTrack, EasyOCR — Concepts : Tracking, OCR, Analytics}

\textbf{Système de gestion du Hadj} \hfill 2024 \\
Plateforme logistique end-to-end pour la gestion de groupes et d’itinéraires. \\
\textit{Tech : React, FastAPI, PostgreSQL — Concepts : Workflow, Synchronisation}

\textbf{POS \& e-commerce hybride} \hfill 2025 \\
Système POS offline-first intégré à une boutique en ligne. \\
\textit{Tech : PyQt, SQLite, PostgreSQL — Concepts : Sync offline/online, Inventory Management}
}

%===========================
\section{LANGUES}
Arabe : Langue maternelle \quad|\quad Français : Courant \quad|\quad Anglais : Courant

\end{document}
