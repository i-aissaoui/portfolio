\documentclass[a4paper,12pt]{article}
\usepackage[utf8]{inputenc}
\usepackage[T1]{fontenc}
\usepackage[german]{babel}
\usepackage{geometry}
\geometry{left=1.8cm,right=1.8cm,top=1.3cm,bottom=1.6cm}

\usepackage{lmodern}
\usepackage{xcolor}

\definecolor{titleblue}{RGB}{0,51,140}

\usepackage{hyperref}
\hypersetup{
    colorlinks=true,
    urlcolor=titleblue,
    linkcolor=titleblue,
    citecolor=titleblue,
    pdfborder={0 0 0}
}

\usepackage{titlesec}
\usepackage{enumitem}

\pagestyle{empty}

\titleformat{\section}
  {\color{titleblue}\Large\bfseries\uppercase}{}{0em}{}
  [\color{titleblue}\titlerule]

\titlespacing\section{0pt}{8pt}{4pt}
\setlist[itemize]{leftmargin=*,topsep=2pt,itemsep=1pt}

\begin{document}

\begin{center}
    {\Huge\bfseries\color{titleblue} AISSAOUI Ismail}\\[4pt]
    {\Large KI- \& Data-Science-Ingenieur}\\[6pt]

    {\normalsize
    E-Mail: \href{mailto:ismail.aissaoui.pro@gmail.com}{ismail.aissaoui.pro@gmail.com} \quad
    Telefon: +213 660 70 77 96 \\
    Portfolio: \href{https://ismail-aissaoui.vercel.app}{ismail-aissaoui.vercel.app} \quad
    Standort: Chlef, Algerien \\
    LinkedIn: \href{https://linkedin.com/in/aissaoui-ismail-6a77a92a7}{linkedin.com/in/aissaoui-ismail-6a77a92a7} \quad
    GitHub: \href{https://github.com/i-aissaoui}{github.com/i-aissaoui}
    }
\end{center}

%===========================
\section{PROFIL}
KI- und Data-Science-Ingenieur mit Spezialisierung auf Transformer-Modelle, NLP, Empfehlungssysteme, Optimierung, Computer Vision und MLOps. Konzipiert und implementiert End-to-End-KI-Lösungen für HR-Tech, Mobilität, Einzelhandel und Bildung. Hohe Expertise in Modelloptimierung, ML-Pipelines und der Ausrichtung technischer Ergebnisse auf geschäftliche Ziele.

%===========================
\section{FÄHIGKEITEN}
\setlength{\parskip}{-3pt}
{\normalsize
\textbf{Programmierung}: Python, TypeScript, JavaScript, Java, SQL \\

\textbf{ML / DL}: PyTorch, TensorFlow, Keras, Scikit-learn, Lightning, ONNX \\

\textbf{KI-Techniken}: Transformers (BERT, GPT, RAG), GNN, GAN, XGBoost, LSTM \\

\textbf{Optimierung}: PSO, Genetischer Algorithmus, ACO, Simulated Annealing, Bayesian Optimization \\

\textbf{Web \& API}: React, Next.js, FastAPI, Node.js, TailwindCSS \\

\textbf{Data \& MLOps}: MLflow, Airflow, Kubeflow, DVC, Pandas, NumPy \\

\textbf{Datenbanken}: PostgreSQL, MongoDB, MySQL, Neo4j, Cassandra \\

\textbf{Cloud \& DevOps}: Docker, Linux, Git, CI/CD, CUDA, AWS, Grafana \\

\textbf{Soft Skills}: Technische Führung, Kommunikation, Stakeholder-Management
}

%===========================
\section{BERUFSERFAHRUNG}
\textbf{Praktikant Netzwerktechnik} \hfill Sep 2024 \\
Algérie Télécom RMC Center — Chlef, Algerien

%===========================
\section{AUSBILDUNG}
\textbf{Nationale Hochschule für Informatik (ESI-SBA)} \hfill 2021--2026 \\
Ingenieur- und Masterstudium in Informatik \\
Spezialisierung: Künstliche Intelligenz \& Informatik

%===========================
\section{WICHTIGE ERFOLGE}
\begin{itemize}
    \item Entwicklung maßgeschneiderter Transformer-Modelle (BERT, GPT, RAG) für reale mehrsprachige Anwendungen.
    \item Aufbau erklärbarer KI-Plattformen für HR-Tech und öffentliche Institutionen zur Reduktion von Bias.
    \item Graphbasiertes Empfehlungssystem mit +24\% Genauigkeitssteigerung (Datensatz: 60\,000 Profile).
    \item Umsetzung vollständiger ML-Pipelines: Datenaufnahme, Validierung, Training, CI/CD und Monitoring.
    \item Echtzeit-Vision-Systeme (YOLO, ByteTrack) mit 32\% reduzierter Latenz.
\end{itemize}

%===========================
\newpage

\section{PROJEKTE}
\setlength{\parskip}{11pt}
{\small

\textbf{GNN-basiertes Recruiting-System (Career Connect)} \hfill 2024--2025 \\
CV-/Job-Matching auf Basis von Graphen und relationalen Embeddings. \\
\textit{Tech: PyTorch Geometric, SBERT, FastAPI, PostgreSQL — Konzepte: GNN, Ähnlichkeit, Embeddings}

\textbf{Aura — KI-Trading-System} \hfill 2025 \\
Full-Stack-Krypto-Trading-Plattform mit LSTM-Prognosen und \textbf{Reinforcement Learning} (Deep Q-Learning) zur Optimierung von Entry-/Exit-Strategien anhand langfristiger Reward-Maximierung. \\
\textit{Tech: TensorFlow, Keras, FastAPI, Next.js, WebSockets — Konzepte: Zeitreihen, RL, DQN}

\textbf{Fortgeschrittene NLP-Pipeline mit Multi-Optimierung} \hfill 2025 \\
Mehrsprachige Moderation optimiert mittels PSO, genetischem Algorithmus und bayesscher Optimierung. \\
\textit{Tech: DistilBERT, PyTorch, FastAPI — Konzepte: Optimierung, Transformer, Klassifikation}

\textbf{Intelligenter Universitätsplaner} \hfill 2025 \\
Automatische Erstellung kollisionsfreier Stundenpläne mithilfe hybrider Heuristiken. \\
\textit{Tech: FastAPI, NumPy, Pandas — Konzepte: GA, PSO, ACO, Simulated Annealing}

\textbf{Mini-GPT (Decoder-only)} \hfill 2025 \\
Implementierung eines kompakten GPT-Modells mit kontrolliertem Training und sequenzieller Textgenerierung. \\
\textit{Tech: PyTorch, CUDA, AdamW, W\&B — Konzepte: Transformer, Language Modeling}

\textbf{North African Sentiment BERT} \hfill 2025 \\
LoRA-Anpassung von BERT für Sentimentanalyse im maghrebinischen Dialekt. \\
\textit{Tech: PyTorch, LoRA, MLflow — Konzepte: Fine-tuning, Low-Rank Adaptation}

\textbf{Geo-RAG — Kartenassistent} \hfill 2025 \\
RAG-System mit geospatialen Daten für kartographische Abfragen. \\
\textit{Tech: ChromaDB, GeoPandas, Streamlit — Konzepte: RAG, Räumliches Denken}

\textbf{Recognizini — Gesichtserkennung} \hfill 2025 \\
Gesichtserkennung mit inkrementellen Embeddings und aktivem Feedback-Mechanismus. \\
\textit{Tech: OpenCV, PyTorch, Redis, FastAPI — Konzepte: Embeddings, Drift-Management}

\textbf{Echtzeit-PPE-Erkennung (YOLOv8)} \hfill 2025 \\
Echtzeit-Erkennung von Schutzausrüstung zur Compliance-Überwachung. \\
\textit{Tech: YOLOv8, OpenCV — Konzepte: Objekterkennung, Streaming}

\textbf{Traffic Analytics Suite} \hfill 2025 \\
Verkehrsanalyse mit Tracking, Geschwindigkeitsmessung und Kennzeichenerkennung. \\
\textit{Tech: YOLOv8, ByteTrack, EasyOCR — Konzepte: Tracking, OCR, Analytik}

\textbf{Hadj-Management-System} \hfill 2024 \\
End-to-End-Logistikplattform für Gruppenmanagement und Reiserouten. \\
\textit{Tech: React, FastAPI, PostgreSQL — Konzepte: Workflow, Synchronisierung}

\textbf{Hybrid-POS- \& E-Commerce-System} \hfill 2025 \\
Offline-first POS-System mit Online-Shop-Integration. \\
\textit{Tech: PyQt, SQLite, PostgreSQL — Konzepte: Offline-/Online-Sync, Inventarverwaltung}
}

%===========================
\section{SPRACHEN}
Arabisch: Muttersprache \quad|\quad Französisch: Fließend \quad|\quad Englisch: Fließend

\end{document}
